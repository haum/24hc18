\documentclass[a4paper, 11pt]{article}

	\usepackage[utf8]{inputenc}
	\usepackage[T1]{fontenc}
	\usepackage[french]{babel}
	\usepackage[most]{tcolorbox}
	\usepackage{xcolor}
	\usepackage{tikz}
	\usepackage{url}
	\usepackage[left=3cm, right=3cm, top=2cm, bottom=2cm]{geometry}

	% fonts
	\usepackage{fontspec}
	\setmainfont{FiraSans-Regular.otf}[
		Path 	       = ./font/,
		BoldFont       = FiraSans-Bold.otf ,
		ItalicFont     = FiraSans-Italic.otf ,
		BoldItalicFont = FiraSans-BoldItalic.otf ]

	% colors
	\definecolor{Sred}{HTML}{DC322F}
	\definecolor{Sblue}{HTML}{268BD2}
	\definecolor{Sgreen}{HTML}{859900}


	\newtcolorbox{warningbox}{ enhanced, colback=Sred!05, coltext=Sred, colframe=Sred }
	\newtcolorbox{tipbox}{ enhanced, colback=Sgreen!05, coltext=Sgreen, colframe=Sgreen }
	\newtcolorbox{infobox}{ enhanced, colback=Sblue!05, coltext=Sblue, colframe=Sblue }

\begin{document}
	\begin{center}
		\LARGE\bfseries
		Marabunta

		\includegraphics[width=8cm]{../documentation/_static/images/logo_marabunta.png}
	\end{center}


	\section{Introduction}

La planète Apocrite est un monde peuplé de fourmis au sein du système
Holométabole. Elle évolue au cœur d'un environnement numérique impitoyable
gouverné par un simulateur intransigeant.

Fraîchement échouée sur la planète, votre colonie de fourmis va devoir faire
preuve d'intelligence collective mais aussi d'autonomie pour survivre dans cet
environnement inconnu. Vous savez qu'Apocrite n'est pas dénuée de ressources
alimentaires mais redoutez la présence d'autres colonies hostiles, même dans
cette banlieue de la galaxie.

En tant que stratège, vous aurez à définir le comportement et la tactique
adoptés par chacune des fourmis et par la fourmilière. Notez qu'aucun dieu n'est
aux commandes dans cet univers et que les orientations de votre colonie ne
dépendront que de la somme des interactions individuelles de chacun de ses
membres avec l'environnement.

Dans cette simulation, la réalité est simplifiée. Vos fourmis disposent donc de
possibilités bornées dont la description complète est faite dans un autre
chapitre. Il n'y a aucun obstacle et les fourmis, dont la mémoire est très
limitée, peuvent communiquer exclusivement en déposant des phéromones.

	\section{Environnement de développement}

Vous \textbf{devez} utiliser un système de gestion de version : Git. Nous nous en
servirons pour récupérer votre code et le compiler/l'interpréter sur nos systèmes.
Vous devez également utiliser un langage parmi :

\begin{itemize}
	\item AsciiDots
	\item Python 2
	\item Python 3
	\item C++
	\item C
	\item C\#
	\item Java
	\item Perl
\end{itemize}

Consultez la documentation pour savoir comment seront compilés les programmes et comment
interagir avec le simulateur :

\begin{center}
	\color{Sblue}\Large\bfseries
	24hc18.haum.org
\end{center}

\textcolor{Sred}{
	\textbf{Note :} Votre code tourne en environnement clos (\textit{jail}). Essayer de
	s'échapper c'est être \textbf{immédiatement} disqualifié. L'environnement complet
	pourra être redémarré à l'envi par le système de jeu pour s'assurer du bon respect des
	consignes (à sa seule discrétion).
}

	\section{Déroulement du jeu}

À intervalle régulier, nous récupérerons le code disponible sur les dépôts Git,
le compilerons pour chacune des équipes et lancerons une série de matches. Le planning des
matches (heure et scénario proposé) sera annoncé à l'avance.


Les matches ont durée finie pré-déterminée en fonction du scénario. Au terme du match des
points sont attribués aux équipes :
\begin{itemize}
	\item Votre code n'est pas récupérable/ne compile pas : \textcolor{Sred}{0 point}.
	\item Votre code permet à la colonie de survivre sans gagner : \textcolor{Sred}{1 point}.
	\item Votre code mène la colonie à la victoire : \textcolor{Sred}{2 points}.
\end{itemize}

\bigskip

La victoire d'une colonie est déterminée en fonction de sa prospérité : somme des
ressources accumulées dans le nid et de la population en vie. Si plusieurs colonies
survivent jusqu'au terme, la plus prospère gagne. En cas d'\textit{ex æquo} les deux
colonies marquent 2 points.

À la lecture de ces précisions, comprenez bien qu'avoir un système permettant à la colonie
de survivre est \textbf{essentiel}. Il n'est pas toujours nécessaire de se battre pour
gagner\ldots

	\section{Victoire, 24\textsuperscript{ème} heure et éclat de lune}

Marabunta est un jeu entre équipes (et parfois contre le monde). Votre objectif en tant
qu'équipe est de marquer plus de points que les autres pour un même nombre de matches.

A la fin des 24h, l'équipe ayant le plus de points gagne, le \textcolor{Sred}{dernier
match ayant lieu 1h avant la fin}. La dernière heure sera consacrée à un temps de
discussion entre les équipes ayant choisi le sujet et celle l'ayant préparé.

\bigskip

\textcolor{Sblue}{Parce que les 24h c'est aussi et surtout un temps de partage entre
passionnés. Les équipes du sujet Marabunta se verront proposer d'aller boire un verre pour
se détendre et faire connaissance au cours de la nuit.}

	\section{Communication}

	Comme l'an dernier, les gens ayant écrit le sujet seront disponibles tout au long du concours pour vous orienter ou
	vous aider. Vous pourrez nous trouver :

	\begin{itemize}
		\item sur IRC : \url{irc.lc/freenode/24hc18}
		\item autour de nos machines
		\item autour d'un café
		\item pas loin d'une blanquette
		\item au babyfoot
	\end{itemize}
\end{document}

