\documentclass[a4paper, 11pt]{article}

	\usepackage[utf8]{inputenc}
	\usepackage[T1]{fontenc}
	\usepackage[french]{babel}
	\usepackage[most]{tcolorbox}
	\usepackage{xcolor}
	\usepackage{tikz}
	\usepackage{url}
	\usepackage[left=3cm, right=3cm, top=2cm, bottom=2cm]{geometry}

	% fonts
	\usepackage{fontspec}
	\setmainfont{FiraSans-Regular.otf}[
		Path 	       = ./font/,
		BoldFont       = FiraSans-Bold.otf ,
		ItalicFont     = FiraSans-Italic.otf ,
		BoldItalicFont = FiraSans-BoldItalic.otf ]

	% colors
	\definecolor{Sred}{HTML}{DC322F}
	\definecolor{Sblue}{HTML}{268BD2}
	\definecolor{Sgreen}{HTML}{859900}

	\setlength{\parskip}{1ex}

	\newtcolorbox{warningbox}{ enhanced, colback=Sred!05, coltext=Sred, colframe=Sred }
	\newtcolorbox{tipbox}{ enhanced, colback=Sgreen!05, coltext=Sgreen, colframe=Sgreen }
	\newtcolorbox{infobox}{ enhanced, colback=Sblue!05, coltext=Sblue, colframe=Sblue }

\begin{document}
	\begin{center}
		\LARGE\bfseries
		Marabunta

		\includegraphics[width=15cm]{../documentation/_static/images/logo_marabunta.png}
	\end{center}


	\section{Introduction}

La planète Apocrite est un monde isolé au sein du système Holométabole peuplé de
fourmis. Elle évolue au cœur d’un environnement numérique impitoyable gouverné
par un simulateur intransigeant.

Fraîchement échouée sur la planète, votre colonie de fourmis va devoir faire
preuve d’intelligence collective mais aussi d’autonomie pour survivre dans cet
environnement inconnu. Vous savez qu’Apocrite n’est pas dénuée de ressources
alimentaires mais redoutez la présence d’autres colonies hostiles, même dans
cette banlieue de la galaxie.

En tant que stratège, vous aurez à définir le comportement et la tactique
adoptés par chacune des fourmis et par la fourmilière. Notez qu’aucun dieu n’est
aux commandes dans cet univers et que les orientations de votre colonie ne
dépendront que de la somme des interactions individuelles de chacun de ses
membres avec l’environnement.

Dans cette simulation, la réalité est simplifiée. Vos fourmis disposent donc de
possibilités bornées dont la description complète est faite dans la
documentation. Il n’y a aucun obstacle et les fourmis, dont la mémoire est très
limitée, peuvent communiquer exclusivement en déposant des phéromones.

	\section{Langages de programmation}

Le code que vous produirez sera compilé et exécuté sur notre serveur de jeu.
Nous limitons notre support aux langages suivants. Vous êtes libres d’utiliser
le langage qui vous convient parmi cette liste :

\begin{itemize}
	\item AsciiDots
	\item C
	\item C++
	\item Java
	\item Perl
	\item Python 2
	\item Python 3
\end{itemize}

	\section{Documentation}

\textcolor{Sblue}{Une documentation détaillée est disponible en ligne. Elle
présente le contexte, les outils et tout ce dont vous avez besoin pour réaliser
le sujet. Nous vous conseillons de bien la lire et de vous y reporter
régulièrement : elle est riche et sera éventuellement complétée selon les
questions des équipes.}

\begin{center}
	\color{Sblue}\Large\bfseries
	marabunta.haum.org
\end{center}

\hfill

	\section{Déroulement}

Le défi s’étale sur une série de matchs se déroulant en plusieurs tournois tout
au long de ces 24h. Ces matchs proposeront des scenarii différents, de
difficulté a priori croissante.

Nous annoncerons les prochains tournois à l’avance. Pour les premiers matchs,
les caractéristiques principales des scenarii à jouer vous seront indiquées.

Les sujets proposés par le HAUM n’ont pas l’habitude d’être à visée scolaire. Ce
sujet-ci ne déroge pas à la règle et est tout à fait dans l’esprit du
hackerspace : ce week-end, nous partageons collectivement une aventure
expérimentale.

Ce qui nous intéresse n’est pas votre capacité à résoudre un problème fermé mais
bien l’ingéniosité que vous mettrez pour imaginer une solution possible à cette
problématique atypique.

\textcolor{Sblue}{Parce que les 24h c'est aussi et surtout un temps de partage
entre passionnés. Les équipes du sujet Marabunta se verront proposer d'aller
boire un verre pour se détendre et faire connaissance au cours de la nuit.}

Plusieurs autres rendez-vous seront organisés pendant la durée de l’épreuve pour
prendre du recul, discuter, se détendre. N’hésitez pas également à venir
bavarder avec l’équipe du sujet et faire part de vos idées ingénieuses ou de vos
difficultés. Nous serons avec vous toute la nuit.

La dernière heure sera consacrée à un temps de discussion entre toutes les
personnes ayant travaillé sur le sujet.

	\section{Nous trouver}

Comme l'an dernier, les gens ayant écrit le sujet seront disponibles
tout au long du concours pour vous orienter ou vous aider. Vous pourrez
nous trouver :

\begin{itemize}
	\item sur IRC : \url{irc.lc/freenode/24hc18}
	\item autour de nos machines
	\item autour d'un café
	\item pas loin d'une blanquette
	\item au babyfoot
\end{itemize}

\end{document}

